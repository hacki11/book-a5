\documentclass{article}

\pagestyle{empty}

\usepackage[
  paperwidth=344mm,
  paperheight=245mm,
  margin=0mm,
  noheadfoot
]{geometry}

\usepackage{graphicx}
\usepackage{tikz}
\usetikzlibrary{calc}
\usepackage{xcolor}

\begin{document}
\begin{tikzpicture}[remember picture,overlay]

  % -------------------------------------------------
  % Datenformat: 344 x 245 mm
  % U4 und U1 + Spine leicht überziehen, aber links bei 0 starten.
  % -------------------------------------------------

  % Rückseite U4 (.jpg) – Start exakt bei x=0, unten minimal überziehen
  % Breite 168mm -> reicht von x=0 bis 168mm (2mm Überlappung mit Spine)
  \node[anchor=south west] at ($(current page.south west)+(-1mm,-1mm)$) {%
    \includegraphics[width=169mm,height=247mm]{u4-image.jpg}%
  };

  % Rücken (Spine) (.jpg) – mittig, mit leichter Überlappung
  \node[anchor=south west] at ($(current page.south west)+(168mm,-1mm)$) {%
    \includegraphics[width=10mm,height=247mm]{spine-image.jpg}%
  };

  % Vorderseite U1 (.jpg) – rechts + unten leicht überziehen
  \node[anchor=south west] at ($(current page.south west)+(178mm,-1mm)$) {%
    \includegraphics[width=168mm,height=247mm]{u1-image.jpg}%
  };

\end{tikzpicture}
\end{document}
